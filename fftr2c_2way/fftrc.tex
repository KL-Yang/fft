\documentclass[a4paper,12pt]{article}
\usepackage{amsmath}
\usepackage{listings}
\lstset{
  language=C,                % choose the language of the code
  numbers=left,              % where to put the line-numbers
  basicstyle=\ttfamily\scriptsize,
  breaklines=true,
}
\usepackage{gnuplot-lua-tikz}
\title{FFT C2C as 2-way R2C Transform}
\author{YangKunlun}
\begin{document}
%There is a similar page describing this
%http://www.engineeringproductivitytools.com/stuff/T0001/PT10.HTM
\maketitle
\section{Parellel Problem}
\begin{figure}
\begin{center} \input{pic/x87_r2c_c2c.tex} \end{center}
    \caption{X87 Instruction Set for Commonly Seen FFT Length} \label{fig:x87}
\end{figure}
\begin{figure}
\begin{center} \input{pic/avx_r2c_c2c.tex} \end{center}
    \caption{AVX Instruction Set for Commonly Seen FFT Length} \label{fig:avx}
\end{figure}
The Surface Related Multiple Elimination (SRME) is a widely used algorithm in seismic processing,
yet it can be simplified into the this convolution based computational model.
\begin{equation}
    M_{a,b} = \sum_{i=-\infty}^{\infty}P_{a,i}\ast P_{i,b}
\end{equation}
The algorithm performs convolution in frequency domain, 
and it makes a lot of forward r2c transforms to produce one result.
The r2c transform is one of the critical region to the optimization of SRME.
\section{SIMD Efficiency}
Cost of r2c transform is half of c2c transform due to symmetry.
On a 2.4G Hz i3-3110M CPU, the TPS (Transforms Per Second) of FFTW library compiled into x87 instruction set 
is shown in Figure \ref{fig:x87}.
The numbers are consistent with the theory about computation costs. 
\par
However, if AVX instruction is enabled, things go differently as shown in Figure \ref{fig:avx}.
This means AVX instruction in FFTW's r2c transform is not as well exploit as in c2c transfrom,
and it gives us the possibility to use c2c transform as a 2-way r2c transform.
\par
The SSE instruction is capable of operating on 4 float numbers a the same time, and AVX is capable of 8,
so ideally, if we care only about throughput,
we should implement 4-way transform for SSE and 8-way transform for AVX to better utilize the hardware's capability.
But before that, let's try 2-way r2c transfrom on top of c2c transform.

\section{2-Way R2C Transform}
Assume we have two discrete real signal $a_j$ and $b_j$, where $j\in [0,N)$,
and the corresponding DFT are $A_j$ and $B_j$.
Since $a_j$ and $b_j$ are real signal, we know that $A_{N-j}=\bar{A_j}$, 
and $B_{N-j}=\bar{B_j}$, where $j=[1,N/2)$. 
    \par
To use the complex algorithm to computer two real transform at one time, 
we can construct a complex signal $c_j=a_j+Ib_j$, and the corresponding DFT is $C_j$.
We also know that $C_j=A_j+IB_j$.
There is a special case for the DC part of the transform that $A_0=C^r_0$, 
and $B_0=C^i_0$, the upper scripts indicate real and image part.
\par
The following equation can be used to solve other frequency component of
the two real transform.
\begin{align}
    C_j &= A_j+IB_j  \\
    C_{N-j} &= A_{N-j}+IB_{N-j} \\
    j &\in [1, N/2) 
\end{align}
\par
The above equations can be further simplified and solve as
\begin{align}
    A_j &= \frac{\bar{C}_{N-j}+C_j}{2} \\
    B_j &= \frac{\bar{C}_{N-j}-C_j}{2}I 
\end{align}

\section{Faster convolution}
For convolution of series $a_j$ and $b_j$, we need to calculate $A_jB_j$.
\begin{equation}
    A_jB_j=\frac{\bar{C}^2_{N-j}-C^2_j}{4}I , j\in[1, N/2) 
        \label{equ:2way}
\end{equation}
\par
The following list is SSE3 intrinsic of two complex multiplications, 
it is more or less the same as what \emph{icc} will generate for vectorization.
\begin{lstlisting}
inline static __m128 cmul_sse3_i(__m128 a, __m128 b)
{
    __m128 c = _mm_mul_ps(_mm_moveldup_ps(a), b);
    b = _mm_shuffle_ps(b, b, _MM_SHUFFLE(2,3,0,1));
    a = _mm_mul_ps(_mm_movehdup_ps(a), b);
    return _mm_addsub_ps(c, a);
}
\end{lstlisting}
Based on the above code, convolution Equation \ref{equ:2way} that use c2c (2-way r2c) transform 
can be implement as shown in the following list.
\begin{lstlisting}
void two_way_conv(const complex float *C, int nr, complex float *D)
{
    int nc = nr/2+1;
    int nn = (nc%2)?(nc):(nc-1);
    D[0] += ((float*)C)[0]*((float*)C)[1]; //DC component

    for(int i=1; i<nn; i=i+2) {
        __m128 ci = _mm_loadu_ps((float*)(C+i));
        __m128 cj = _mm_mul_ps(_mm_loadu_ps((float*)(C+nr-1-i)), 
               /*conjugate*/   _mm_set_ps(-1.0f, 1.0f, -1.0f, 1.0f)); 
        cj = _mm_shuffle_ps(cj, cj, _MM_SHUFFLE(1,0,3,2));
        __m128 pl = _mm_add_ps(cj, ci);
        __m128 mi = _mm_sub_ps(cj, ci);
        __m128 xx = cmul_sse3_i(pl, mi); //multiple (a+b)(a-b)
        xx = _mm_shuffle_ps(xx, xx, _MM_SHUFFLE(2,3,0,1));
        xx = _mm_mul_ps(xx, _mm_set_ps(0.25f, -0.25f, 0.25f, -0.25f));
        xx = _mm_add_ps(xx, _mm_loadu_ps((float*)(D+i)));   //+=
        _mm_storeu_ps((float*)(D+i), xx); //multiple by I/4
    }
    if(nn!=nc) {
        complex float A =   (conjf(C[nr-nn])+C[nn])/2.0f;
        complex float B = I*(conjf(C[nr-nn])-C[nn])/2.0f;
        D[nn] += A*B;
    }
}
\end{lstlisting}
\par
I need to consider how to implement this use AVX for this 2-way r2c transform.
But for better performance, 4-way or 8-way should be a better choice, 
maybe I should try OpenCL.
\section{Conclusion}
Benchmark of one example using c2c as 2-way r2c transform for fast convolution. 
I squeeze about 30\% more performance out of FFTW, that is not an easy task.
However, when the transform length goes to 4096, the boost drops to about 16\%. 
And more or less consistant with trend in Figure \ref{fig:avx}.
The source code that generates the below result and validates the result should come along with this document.
\begin{verbatim}
start testing speed....
nr=1536, nc=769 step=1536
FFT of 1536x28800 repeat 20 times:
  fftw : 4968.310059 ms
  newf : 3776.236084 ms
\end{verbatim}
%\section{TODO}
%Put all these work on google code, under my project!
\end{document}
