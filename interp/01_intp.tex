\title{Interpolation using deritive}
\author{KL-Yang}
\date{\today}

\documentclass[12pt]{article}
\usepackage{amsmath}
\usepackage{amssymb}

\begin{document}
\maketitle

\begin{abstract}
This is a note of how to setup the interpolation equations in Fourier domain utilizing derivative samples.
The problem is a concept abstraction of the method used in WesternGeco's isometrix seismic acquistion.
The equations are derived or picked from many references.
\end{abstract}

\section{Matrix form of DFT}
The DFT (Discreted Fourier Transform) is defined as:
\begin{equation}
    X_n = \sum_{n=0}^{N-1} x_n e ^ {-i2\pi kn/N}
\end{equation}
Which in matrix vector multiplication form is:
\begin{equation}
\begin{bmatrix}
    X_0 \\ X_1 \\ X_2 \\ X_3 \\ \vdots \\ X_{N-1} 
\end{bmatrix}
=
    \begin{bmatrix}
        1 & 1 & 1 & 1 & \ldots & 1 \\
        1 & \omega & \omega^2 & \omega^3 & \ldots & \omega^{N-1} \\
        1 & \omega^2 & \omega^4 & \omega^6 & \ldots & \omega^{2(N-1)} \\
        1 & \omega^3 & \omega^6 & \omega^9 & \ldots & \omega^{3(N-1)} \\
        \vdots & \vdots & \vdots & \vdots & \ddots & \vdots \\
        1 & \omega^{N-1} & \omega^{2(N-1)} & \omega^{3(N-1)} & \ldots & \omega^{(N-1)(N-1)} 
    \end{bmatrix}
    \begin{bmatrix}
        x_0 \\ x_1 \\ x_2 \\ x_3 \\ \vdots \\ x_{N-1}
    \end{bmatrix}
\end{equation}
Where $\omega=e^{-2\pi i/N}$, be careful, $\omega$ may denote to different thing in this note,
it will be explained explicitly everytime.
If we scale the matrix by $1/\sqrt{N}$ and write it as $W_t$, the equation in short form:
\begin{equation}
    \bar{X} = W_t\bar{x}
\end{equation}
For inverse DFT, the appearance of the matrix is the same, except that $\omega=e^{2\pi i/N}$, 
we write it as $W_t^{-1}$, $\bar{x} = W_t^{-1}\bar{X}$.

\section{Fourier transform of derivative}

Given the Fourier transform of a general function, its derivative is:
\begin{eqnarray}
    F.T[f(t)]  &=& F(\omega)   \\
    F.T[f'(t)] &=& i\omega F(\omega)
\end{eqnarray}
For DFT, a continous function $f(n)$ can be found to satisfy:
\begin{eqnarray}
    x_n  &=& f(n)    \\
    X_n  &=& F.T[f(t)] = F(\omega)|_{n \to \omega} \\
    x'_n &=& f'(n) \\
    X'_n &=& F.T[f'(t)] = i\omega F(\omega)|_{n\to \omega} = i\omega X_n|_{n\to \omega} 
\end{eqnarray}
The trigonometric interpolation polynomial can be such a function! And here the $\omega$ is 
angle frequency from $-2\pi f_{nyquist}$ to $2\pi f_{nyquist}$.


\section{Matrix form of derivative}
Considering the trigonometric interpolation polynomial. For real number $x_n$ of even length $N$
\footnote{Even and odd shall be handled differently}, the derivative $x'_n$:
\begin{eqnarray}
    \bar{x'} & = & W_t^{-1} D \bar{X} \\
    D &=&
    \begin{bmatrix}
        0 & 0 & 0 & \ldots & 0 & 0 & 0 & \ldots & 0 \\
        0 & 2f & 0 & \ldots & 0 & 0 & 0 & \ldots & 0 \\
        0 & 0 & 4f & \ldots & 0 & 0 & 0 & \ldots & 0 \\
        \vdots & \vdots & \vdots & \ddots & \vdots & \vdots & \vdots & \vdots & \vdots \\
        \vdots & \vdots & \vdots & \ldots & (N-2)f & 0 & 0 & \ldots & 0 \\
        \vdots & \vdots & \vdots & \ldots & 0 & Nf & 0 & \ldots & 0 \\
        \vdots & \vdots & \vdots & \ldots & 0 & 0 & -(N-2)f & \ldots & 0 \\
        \vdots & \vdots & \vdots & \ldots & \vdots & \vdots & \vdots & \ddots & \vdots \\
        0 & 0 & 0 & \ldots & 0 & 0 & 0 & \ldots & -2f \\
    \end{bmatrix}
\end{eqnarray}
where $f=i2\pi f_{nyquist}/N$, and $D$ is a diagonal matrix
\footnote{Some reference on Google gives wrong result about this!}.

\section{Definition of the problem}
The known are $N$ measurement of samples $x_n$ and the derivative $x'_n$.
The interpolation need to utilize both measurement 
to extend the frequency to the range from $0$ to $2f_{nyquist}$.

Assume $N$ is even number. If $N$ is odd, equations shall be formed differently. 
The interpolation need to be able to unfold alias, otherwise would be useless. 
To be realistic, it has to be assumed that:
\begin{enumerate}
    \item Original sample $\bar{x}_n$ is aliased.
    \item Signal is band limited to $2f_{nyquist}$.
\end{enumerate}
Fail to satisfy the assumptions will render the interpolation meaningless.

\subsection{First equation}
Let the interpolated signal be $y_n$, and its spectrum be $Y_n$.
\begin{equation}
    \begin{bmatrix}
        x_0 \\ \varnothing \\ x_1 \\ \vdots \\ \varnothing \\ x_{N/2} \\ \varnothing \\ \vdots \\ x_N \\ \varnothing
    \end{bmatrix}
\Leftrightarrow
    \begin{bmatrix}
        y_0 \\ y_1 \\ y_2 \\ \vdots \\ y_{N-1} \\ y_N \\ y_{N+1} \\ \vdots \\ y_{2N-2} \\ y_{2N-1}
    \end{bmatrix}
= W_t^{-1}|_{2N\times 2N}
    \begin{bmatrix}
        Y_0 \\ Y_1 \\ Y_2 \\ \vdots \\ Y_{N-1} \\Y_N \\ Y_{N+1} \\ \vdots \\ Y_{2N-2} \\ Y_{2N-1} 
    \end{bmatrix}
\Leftrightarrow
    \begin{bmatrix}
        X_0 \\ X_1  \\ X_2 \\ \vdots \\ X_{N-1} \\ \varnothing \\ \varnothing \\ \vdots \\ \varnothing \\ \varnothing
    \end{bmatrix}
\end{equation}
Where $W_t|_{2N \times 2N}$ is the DFT matrix:
\begin{equation}
W_t|_{2N \times 2N} =
 \begin{bmatrix}
        1 & 1 & 1 & 1 & \ldots & 1 \\
        1 & \omega & \omega^2 & \omega^3 & \ldots & \omega^{2N-1} \\
        1 & \omega^2 & \omega^4 & \omega^6 & \ldots & \omega^{2(2N-1)} \\
        1 & \omega^3 & \omega^6 & \omega^9 & \ldots & \omega^{3(2N-1)} \\
        \vdots & \vdots & \vdots & \vdots & \ddots & \vdots \\
        1 & \omega^{2N-1} & \omega^{2(2N-1)} & \omega^{3(2N-1)} & \ldots & \omega^{(2N-1)(2N-1)} 
    \end{bmatrix}
\end{equation}
It forms $N$ equations, and the DFT matrix decimated to $W_t|_{N\times 2N}$. 
Being alone they are under determined, 
and there is no constrain over $Y_N \to Y_{2N-1}$, 
which is between $f_{nyquist}$ and $2f_{nyquist}$. 


The least sequare solution to these equations lead to frequency domain interpolation, 
where $Y_n$ above $f_{nyquist}$ is simply set to $0$.

\subsection{Second equation}
\begin{equation}
    \begin{bmatrix}
        x'_0 \\ \varnothing \\ x'_1 \\ \vdots \\ \varnothing \\ x'_{N/2} \\ \varnothing \\ \vdots \\ x'_N \\ \varnothing
    \end{bmatrix}
\Leftrightarrow
    \begin{bmatrix}
        y'_0 \\ y'_1 \\ y'_2 \\ \vdots \\ y'_{N-1} \\ y'_N \\ y'_{N+1} \\ \vdots \\ y'_{2N-2} \\ y'_{2N-1}
    \end{bmatrix}
= W_t^{-1}|_{2N\times 2N} D
    \begin{bmatrix}
        Y_0 \\ Y_1 \\ Y_2 \\ \vdots \\ Y_{N-1} \\Y_N \\ Y_{N+1} \\ \vdots \\ Y_{2N-2} \\ Y_{2N-1} 
    \end{bmatrix}
\end{equation}
Where $D$ is previously mentioned diagonal matrix. 
This can form another $N$ equations, with the matrix decimated to $N\times 2N$.

\subsection{Joint equation}
Put the two set of equations together.
\begin{equation}
    \begin{bmatrix}
        \bar{x}~|_{N\times 1} \\
        \bar{x'}|_{N\times 1}
    \end{bmatrix}
    =
    \begin{bmatrix}
        W_t^{-1}~~~|_{N\times 2N} \\
        W_t^{-1} D|_{N\times 2N}  
    \end{bmatrix}
    \begin{bmatrix}
        \bar{Y}|_{2N\times 1}
    \end{bmatrix}
\end{equation}

If the matrix is full rank. The solution will unwrap the alias and expand the frequency to $2f_{nyquist}$.

\section{Solver}
To solve the equation $Ax=b$ use CG.
\begin{eqnarray}
    A &=&
    \begin{bmatrix}
        W_t|_{2N\times N} ~~~ DW_t|_{2N\times N}
    \end{bmatrix}
    \begin{bmatrix}
        W_t^{-1}~~~|_{N\times 2N} \\
        W_t^{-1} D|_{N\times 2N}  
    \end{bmatrix} \\
    x &=&
    \begin{bmatrix}
        \bar{Y}|_{2N\times 1}
    \end{bmatrix} \\
    b &=&
    \begin{bmatrix}
        W_t|_{2N\times N} ~~~ DW_t|_{2N\times N}
    \end{bmatrix}
    \begin{bmatrix}
        \bar{x} |_{N\times 1} \\
        \bar{x'}|_{N\times 1}
    \end{bmatrix}
\end{eqnarray}
For Conjugate Grad




\end{document}
